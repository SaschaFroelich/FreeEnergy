\documentclass[ %
english, % Sprache
12pt, % Standardschriftgröße
a4paper, % Papierformat
titlepage, % Titelseite aktivieren
twoside=on, % Zweiseitiges Layout
numbers=noenddot, % Überschriftennummerierung ohne Punkt am Ende
bibliography=totoc, % Literaturverzeichnis im Inhaltsverzeichnis aufführen
chapterprefix=true, % Kapitel steht vor \chapter-Befehl
open=right % Kapitel auf ungerader Seitenzahl beginnen
]{scrreprt} % Dokumenteigenschaften
\usepackage[utf8]{inputenc} % wie Editoreingaben umwandeln (Umlaute, Sonderzeichen, ...)
\usepackage{float}
\usepackage{mathtools}
\usepackage{ragged2e} %Für text Blocksatz
\usepackage{babel} % Sprachpakete
\usepackage[T1]{fontenc} % Encoder für Font
\usepackage{amsmath,amsthm,amssymb} % Zum Einbinden von Formeln
\usepackage{amssymb} % Zum Einbinden von Symbolen
\usepackage{color} % Für Farben
\usepackage{wrapfig} %To wrap text around figures
\usepackage{afterpage} %Make stuff appear on next page
\usepackage{calc} % Rechenoperationen innerhalb Befehlsdefinitionen
\usepackage[ %
automark, % Automatische Labelsetzung für Kopfzeile
headsepline % Trennstrich zwischen Kopfzeile und Text
]{scrpage2} % Für eigene Kopf- und Fußzeile 
\usepackage{graphicx} % Zum Einfügen von Grafiken
\usepackage{natbib}
\usepackage{graphics}
\usepackage{mathptmx} % Standardschrift ist Times New Roman (auch in Formeln)
\usepackage[rightcaption]{sidecap} % Überschrift neben Bild
\usepackage[ %
format=plain, % Beschriftung ist gewöhnlicher Absatz
justification=justified, % Blocksatz
font=small, % Schriftgröße
labelfont=bf, % Fettes Label
labelsep=colon, % Doppelpunkt als Trenner
figurename=Figure, % Labelname für Bild
tablename=Table % Labelname für Tabelle
]{caption} % Für Bildunterschriften
\usepackage[hypcap=true,subrefformat=simple]{subcaption} % Mehrere Bilder nebeneinander mit Unterschrift
%\usepackage{titlesec} % Überschrifen formatieren
\usepackage[separate-uncertainty]{siunitx} % Für Einheiten
\usepackage{booktabs} % Tabellen besser formatieren können, für ExcelToLatex Makro
\usepackage{tikz,pgfplots} %For graphs and plots
\usepgfplotslibrary{groupplots}
\pgfplotsset{compat=1.6}

\usetikzlibrary{positioning} 
%\usepackage{multirow} % Tabellen besser formatieren können, für ExcelToLatex Makro
%\usepackage{bigstrut} % Tabellen besser formatieren können, für ExcelToLatex Makro
%\usepackage{enumitem} % Für spezielle Aufzählungen
\usepackage[version=3]{mhchem} % Für chemische Formeln

\usepackage[top=4.5cm,bottom=6cm,left=2.75cm,right=3cm,footskip=1cm]{geometry} % Layout
\usepackage[ %
bookmarks=true, % Zeige Lesezeichenleiste
pdftitle={Distinct propagation of brain activity patterns for aversive emotions with and without cognitive regulation}, % Titel
pdfauthor={Sascha Frölich}, % Autor
colorlinks=true, % Umrandete (Box) oder farbige Links
linkcolor=black, % interne Linkfarbe
urlcolor=blue, % Farbe von URL links
citecolor=black % Farbe für Zitierungslinks
]{hyperref} % Links in Pdf - Paket muss als letztes geladen werden!!!
%
% Format von Bildreferenz: \ref{} --> 1.1(a), \subref{} --> (a)
\renewcommand\thesubfigure{(\alph{subfigure})}
%
% Schriftart setzen (Einheitlich Times New Roman) + Überschriften fett:
\setkomafont{disposition}{\bfseries}
%
% Anzahl Ebenen im Inhaltsverzeichnis
\setcounter{tocdepth}{2}
%
% Überschriften formatieren:
%\titlespacing*{\section}{0pt}{*3}{*1.5} % Abstand vor und nach Überschrift
%\titleformat{\section}[hang]{\Large\bfseries}{\makebox[1.5cm][l]{\thesection}}{0pt}{}
%\titleformat{\subsection}[hang]{\large\bfseries}{\makebox[1.5cm][l]{\thesubsection}}{0pt}{}
%\titleformat{\subsubsection}[hang]{\normalsize\bfseries}{\makebox[1cm][l]{\thesubsubsection}}{0pt}{}
%
% Befehl für Referenzen auf Formeln, mit \eref{...} aufrufen
\newcommand{\eref}[1]{Eq.~\eqref{#1}}
%
% Befehl für Referenzen auf Bilder, mit \fref{...} aufrufen
\newcommand{\fref}[1]{Fig.~(\ref{#1})}
%
% Befehl für Referenzen auf Tabellen, mit \tref{...} aufrufen
\newcommand{\tref}[1]{Tab.~\ref{#1}}
%
% Befehl für Referenz  auf einen Abschnitt, mit \sref{...} aufrufen
\newcommand{\sref}[1]{Sec.~\ref{#1}}

\setlength\parskip{\baselineskip}
\usepackage[section]{placeins} % Place floats in the section they are in